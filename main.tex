\newcommand{\subject}[1]{\renewcommand{\subject}{#1}}
\newcommand{\keywords}[1]{\renewcommand{\keywords}{#1}}

\documentclass[a4paper]{article}

\date{\today}
\subject{LaTeX}
\keywords{PDF, DVI.}

\usepackage{docstyle}
\usepackage{multicol}
\usepackage{lipsum}
\usepackage{authblk}
\usepackage{caption}
\usepackage[version=4]{mhchem}

\title{Spin Casting to Confirm the Molecular Weight $M_w$ of a Polystyrene Sample}

\author{Samuel Coopersmith}
\affil{Casa Grande High School}

\author{Brinley Dai}
\affil{}

\author{Sahana Dhama}
\affil{The Wheatley School}

\author{Peter Elmer}
\affil{High School for Math, Science and Engineering}

\author{Dylan Fei}
\affil{Jericho Senior High School}

\author{Hannah Feng}
\affil{Torrey Pines High School}

\author{Anita Gaenko}
\affil{Huron High School}

\author{Jerry Gao}
\affil{Beijing No.~80 High School}

\author{Jacqueline Han}
\affil{}

\author{Anna Heimowitz}
\affil{Stella K. Abraham High School}

\author{Sharis Hsu}
\affil{Valley Christian High School}

\author{Ellen Hu}
\affil{C.~Leon King High School}

\date{}

\newenvironment{Figure}
  {\par\medskip\noindent\minipage{\linewidth}}
  {\endminipage\par\medskip}

\newcommand{\sectionattrib}[1]{\vspace{-0.3cm}\noindent\small\textbf{#1}}

\begin{document}
	\maketitle
    \begin{abstract}
        Molecular weight is crucial to characterizing and understanding polymers. Using the molecular weight, we can approximate the length of the polymer chains in a polystyrene (PS) sample, thereby allowing us to predict properties of the polymer. In this experiment, a method of measuring molecular weight involving spin casting and ellipsometry to confirm a known pure PS sample (Mw=280K) was applied and tested. Employing fourier-transform infrared spectroscopy (FTIR), differential scanning calorimetry (DSC), UV-visible spectrometry, and goniometry, the sample was confirmed to be PS. FTIR analysis confirmed that the sample was pure PS, while DSC determined the glass transition temperature to be 107.41°C, in agreement with reported PS values. 
        
        The sample was then dissolved in various concentrations ranging from 25 mg/mL to 37.5 mg/mL and spun cast on silicon wafer squares. The resulting PS-coated silicon wafers were measured via ellipsometry to determine thickness. Contact angles were determined via a goniometer for both PS-coated (Mean: 82.52°) and non-coated (15.12°) wafers. PS-coated wafers were then exposed to UV/Ozone, and the contact angles were measured, corroborating industrial processes of creating hydrophilic PS materials. 
        
        The findings support previous experiments that use spin-casting to infer the molecular weight of polystyrene samples.
    \end{abstract}
    \newpage
    \begin{multicols}{2}
        \section{Section Name}
        \sectionattrib{Jerry Gao}
        \begin{Figure}
            \centering
            \includegraphics[scale=0.5]{example-image}
            \captionof{figure}{Caption of the figure.}
        \end{Figure}
        \lipsum[3-5]
        \section{Section Name}
        \lipsum[3-5]
        \section{Section Name}
        \begin{Figure}
            \centering
            \includegraphics[scale=0.5]{example-image}
            \captionof{figure}{Caption of the figure.}
        \end{Figure}
        \lipsum[3-5]
        \begin{Figure}
            \centering
            \includegraphics[scale=0.5]{example-image}
            \captionof{figure}{Caption of the figure.}
        \end{Figure}
        \section{Section Name}
        \lipsum[3-5]
        \section{Section Name}
        \lipsum[3-5]
        \begin{Figure}
            \centering
            \includegraphics[scale=0.5]{example-image}
            \captionof{figure}{Caption of the figure.}
        \end{Figure}
        \lipsum[3-5]
        \section{Section Name}
        \lipsum[3-5]
        \section{Section Name}
        \lipsum[3-5]                            
    \end{multicols}
    
\end{document}
